% !TEX program = xelatex
\documentclass{scrartcl}

\usepackage{amsmath}
\usepackage{amssymb}
\usepackage{physics}

\usepackage{hyperref}
\usepackage{cleveref}

% linear map
\newcommand{\linmap}{\xrightarrow{\raisebox{-0.7ex}[0ex][0ex]{$\sim$}}}

\begin{document}

\section{Port-Hamiltonian system without dissipation}

\noindent
flows:
\begin{equation*}
	f(t) := -\dot{x}(t)
\end{equation*}

\noindent
efforts:
\begin{equation*}
	e(t) := \nabla H \bigl( x(t) \bigr)
\end{equation*}

\noindent
dynamics:
\begin{equation*}
	-f(t) = J \bigl( x(t) \bigr) \, e(t) + G \bigl( x(t) \bigr) \, u(t)
\end{equation*}
with inputs $u(t)$ \\

\noindent
outputs:
\begin{equation*}
	y(t) = G^\mathrm{T} \bigl( x(t) \bigr) \, e(t)
\end{equation*}


\section{Continuous collocation methods}

\subsection{Introduction}

Solving the initial value problem
\begin{subequations}
	\begin{alignat}{1}
		\dot{x}(t) &= J \bigl( x(t) \bigr) \, e(t) + G \bigl( x(t) \bigr) \, u(t)
		\label{eq:ph_dynamics} \\
		x(0) &= x_0
		\label{eq:initial_condition}
	\end{alignat}
\end{subequations}
on some time interval $I = t \in \left[ 0, t_f \right]$
is in general not possible analytically. \\

The problem can be solved numerically
by splitting the time interval $I$ into $K$ subintervals
\begin{equation*}
	I^k :=
	\left[ \left( k-1 \right) h, \, k h \right]
	\,.
	\label{eq:subinterval}
\end{equation*}
The index $k = 1, 2, \ldots, K$ identifies a paricular subinterval.
The length of the subintervals has to be prescribed by chosing a time step $h$
such that $t_f = h K$. \\

The true solution $x(t)$ (with $t \in I$) is then approximated
by the numerical solution $\tilde{x}(t)$ which is defined picewise:
On each subinterval,
$\tilde{x}(t)$ (with $t \in I^k$) is a polynomial of degree $s$ (in the variable $t$).
More precisely,
if $x(t) \in \mathbb{R}^N$ then $\tilde{x}(t)$ is a vector of $N$ polynomials. \\

Like the time step $h$,
the oder of the polynomial approximation is a parameter
that has to be chosen by the user (to control the fidelity of the numerical method). \\

Since a polynomial of degree $s$ is determined by $s+1$ coefficients
there are $s+1$ (vectorial) unknowns per interval.
The continuity of the solution between two intervals determines one (vectorial) unknown.
For the first interval $I^1$,
this unknown is determined by the initial condition~\eqref{eq:initial_condition}.

The remaining $s$ unknowns are fixed by requiring that
the differential equation has to hold
at $s$ collocation points $t_1^k, \ldots, t_s^k \in I^k$.
The choice of these points within each subinterval is important
for obtaining a good numerical approximation.


\subsection{Details}

By defining
$t_0^k := \left( k-1 \right) h$ and $t_{s+1}^k := hk$,
each subinterval can be written as
\begin{equation*}
	I_k = \left[ t_0^k, t_{s+1}^k \right]
	\,.
\end{equation*}
Then, the collocation points $t_1^k, \ldots, t_s^k$ can be written as
$t_i^k := t_0^k + c_i \, h$
with coefficients $c_1, \ldots, c_s$
satisfying
\begin{align*}
	&\forall \, i \in \left\{ 1, \ldots, s \right\} \: \colon \: 0 \leq c_i \leq 1 \\
	&\forall \, i \in \left\{ 1, \ldots, s-1 \right\} \: \colon \: c_i < c_{i+1} \,.
\end{align*}

Since on each subinterval the numerical solution $\tilde{x}(t)$
is given as a vector of polynomials of degree $s$,
the time derivative $\dot{\tilde{x}}(t)$ is a vector of polynomials of degree $s-1$. \\

The numerical solution $\tilde{x}(t)$ with $t \in I^k$ can be written as
\begin{equation}
	\tilde{x}(t) =
	\tilde{x}(t_0^k) + \int_{t_0^k}^t \dot{\tilde{x}}(u) \, \mathrm{d}u
	\,.
	\label{eq:numerical_solution}
\end{equation}

To make life easier,
a local time coordinate $\tau^k \in \left[ 0, 1 \right]$ is defined
on each subinterval $I^k$ by requiring that
$t = \bigl( \left( k-1 \right) + \tau \bigr) h$.
Hence, $\tau^k = 0$ corresponds to $t_0^k$
and $\tau^k = 1$ corresponds to $t_{s+1}^k$.

Based on this normalized time,
the polynomial approximation of $\dot{\tilde{x}}$ can be defined
in a unified manner for all $k$:
On each subinterval,
the vector $\dot{\tilde{x}}$ can be written as a linear combination
\begin{equation*}
	\dot{\tilde{x}}(\tau^k) :=
	\sum\limits_{i = 1}^s \Bigl( -f_i^k \, l_i(\tau) \Bigr)
	\,.
	\label{eq:polynomial_approximation_xdot}
\end{equation*}
More precisely,
the $n$-th component of $\dot{\tilde{x}}$ is a vector in the vector space
that is spanned by the $s$ basis functions $l_1, \ldots, l_s$.
Given such a basis,
this vector is determined by the tuple consisting of
the $n$-th component of each of the vectors $f_1, \ldots, f_s$. \\

With this change of time coordinate and a choice of basis,
equation~\eqref{eq:numerical_solution} can be written as
\begin{equation*}
	\begin{alignedat}{1}
		\tilde{x}(t_0^k + \tau^k \, h)
		&= \tilde{x}(t_0^k) +
		\int_0^\tau  \sum\limits_{j=1}^s \Bigl( -f_j^k \, l_j(\mu) \Bigr)
		\, h \, \mathrm{d} \mu \\
		&= \tilde{x}(t_0^k) - h \, \sum\limits_{j=1}^s
		\biggl( f_j^k \, \int_0^\tau l_j(\mu) \, \mathrm{d} \mu \biggr)
		\,.
	\end{alignedat}
\end{equation*}

The collocation method is based on relating $\dot{\tilde{x}}$
and $\tilde{x}$ at the $s$ collocation points $c_1, \ldots, c_s$ (in normalized time).
Hence, it is convenient to chose the basis polynomials such that
the $n$-th component of $f_i$
expresses the value of the $n$-th component of $\dot{\tilde{x}}$ at $\tau = c_i$,
i.e. $-f_i \overset{!}{=} \dot{\tilde{x}}(c_i)$.

These basis polynomials are the Lagrange polynomials
\begin{equation*}
	l_i(\tau) :=
	\prod\limits_{\substack{j = 1 \\ j \neq i}}^s \frac{\tau - c_j}{c_i - c_j}
	\label{eq:Lagrange_polynomial}
\end{equation*}
which have the desired property $l_i(c_j) = \delta_{ij}$ (Kronecker delta). \\

The value of the numerical solution $\tilde{x}$
at the beginning of each subinterval $I^k$
shall be denoted by $x_0^k := \tilde{x}(t_0^k)$.
This value is known because of the continuity of the solution between subintervals
or because of the initial condition~\eqref{eq:initial_condition},
i.e. $x_0^1 = x_0$ and $x_0^{k} = x_{s+1}^{k-1}$ for $k > 1$. \\

By defining
\begin{equation*}
	b_j := \int_0^1 l_j(\mu) \, \mathrm{d} \mu
	\label{eq:butcher_b}
\end{equation*}
the numerical solution at the endpoint of a subinterval can be written as
\begin{equation*}
	x_{s+1}^k := \tilde{x}(t_{s+1}^k) =
	x_0^k - h \, \sum\limits_{j=1}^s \Bigl( b_j \, f_j^k \Bigr)
	\,.
\end{equation*}
It is the main goal of all computation done at a particular step $k$
to obtain the value $x_{s+1}^k$
which is determined by the result $x_{s+1}^{k-1} = x_0^k$ of the previous step
and the coefficients $f_1^k, \, \ldots, f_s^k$.

These coefficients have to be determined by solving a (nonlinear) system of equations:
\Cref{eq:ph_dynamics} has to hold at the collocation points $t_1^k, \ldots, t_s^k$.
Hence, the system of equation defines a relationship between
the values of the derivative of the numerical solution at the collocation points
given by the the unknowns $f_1^k, \ldots, f_s^k$
and the values of the numerical solution
at the collocation points $\tilde{x}(t_1^k), \ldots, \tilde{x}(t_s^k)$
which in turn also depend on $f_1^k, \ldots, f_s^k$. \\

By defining
\begin{equation*}
	a_{ij} := \int_0^{c_i} l_j(\mu) \, \mathrm{d} \mu
	\label{eq:butcher_a}
\end{equation*}
the numerical solution $\tilde{x}$ can conveniently be expressed
at the collocation points:
\begin{equation*}
	x_i^k := \tilde{x}(t_i^k) =
	x_0^k - h \, \sum\limits_{j=1}^s \Bigl( a_{ij} \, f_j^k \Bigr)
\end{equation*}
for $i = 1, \ldots, s$. \\

The following definitions are introduced for convenience:
\begin{subequations}
	\begin{alignat*}{1}
		J_i^k &:= J(x_i^k) \\
		e_i^k &:= \left. \nabla H(x) \right|_{x = x_i^k} \\
		G_i^k &:= G(x_i^k) \\
		u_i^k &:= u(t_i^k)
	\end{alignat*}
\end{subequations}

Then, the system of equations,
which has to be solved at the $k$-th step,
can be written as
\begin{equation}
	f_i^k + J_i^k \, e_i^k + G_i^k \, u_i^k = 0
	\quad \text{for } i = 1, \ldots, s \,.
	\label{eq:gl_system1}
\end{equation}
In the general case, every term of the above equation,
except for $u_i^k$,
depends on (a subset of) the unknowns $f_1^k, \ldots, f_s^k$. \\

By introducing the block vectors and matrices
\begin{subequations}
	\begin{alignat*}{1}
		f^k &:= {\left[ {\left( f_1^k \right)}^\mathrm{T}, \ldots,
			{\left( f_s^k \right)}^\mathrm{T} \right]}^\mathrm{T} \\
		J^k &:= \mathrm{blockdiag}(J_1^k, \ldots, J_s^k) \\
		e^k &:= {\left[ {\left( e_1^k \right)}^\mathrm{T}, \ldots,
			{\left( e_s^k \right)}^\mathrm{T} \right]}^\mathrm{T} \\
		G^k &:= \mathrm{blockdiag}(G_1^k, \ldots, G_s^k) \\
		u^k &:= {\left[ {\left( u_1^k \right)}^\mathrm{T}, \ldots,
			{\left( u_s^k \right)}^\mathrm{T} \right]}^\mathrm{T}
	\end{alignat*}
\end{subequations}
the system of equations~\eqref{eq:gl_system1} can be written as
\begin{equation*}
	f^k + J^k \, e^k + G^k \, u^k = 0
\end{equation*}


\subsection{Gauss-Legendre collocation}

Gauss-Legendre collocation methods use the roots of the shifted Legendre polynomial
\begin{equation*}
	\frac{1}{s!} \frac{\mathrm{d}^s}{\mathrm{d} x^s} \Bigl( {\bigl( x \left( x - 1 \right) \bigr)}^s \Bigr)
\end{equation*}
as coefficients $c_1, \,\ldots, c_s$.



\end{document}
